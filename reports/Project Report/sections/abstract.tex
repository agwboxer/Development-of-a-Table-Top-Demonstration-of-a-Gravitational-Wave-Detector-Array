\font\myfont=cmbx12 at 20pt
\Large
{\myfont Abstract}

Gravitational wave (GW) detectors have given humanity insight into the most extreme cosmic events in the universe. They are the foundational backbone of our understanding of the existence of black holes and neutron stars. The detection of GWs was ground breaking in itself for proving Einstein's theory of relativity correct. Multiple detectors are required to be able to measure the distance between the Earth and the source of the gravitational wave leading to GW detectors being constructed across the globe.

This report explores how multiple observatories collaborate to identify the location of the source of GWs or by extension other cosmic objects. This project aims to develop a tabletop demonstration of a network of GW detectors to explore the principles of source localisation and triangulation. By using multiple water level sensors as analogues for GW detectors, the demonstration will simulate a gravitational wave propagating through a medium (in this instance a water wave) from a singular source. A Raspberry Pi will serve as the central control hub, interfacing with the sensors through custom designed circuitry to measure and process signals. A python program will be developed to analyse the sensor data and infer the position of the source based on time of arrival differences across the network of sensors. This project offers an accessible and engaging way to illustrate the functionality of a global modern GW detector network.