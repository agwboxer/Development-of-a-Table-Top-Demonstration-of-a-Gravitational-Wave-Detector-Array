\section{Summary of Findings}
This project has successfully developed a tabletop demonstration of a GW detector array using water level sensors as analogues for real world GW detectors. The experiment utilises the principles of source localisation and triangulation by simulating GW detection with the aid of ML models, as demonstrated through the hyperbolic equations (Equations \ref{eqn:dd12}-\ref{eqn:dd23}) and iterative optimisation (Equation \ref{eqn:SF}) used to determine the source position \cite{boxer_2025_15041819}. The aim of this project was targeted towards providing an accessible and intuitive demonstration of GW detection principles, suitable for public engagement and educational purposes.

The TDoA method proved effective in localising the source of waves, utilising hyperbolic triangulation (Figures \ref{fig:triangulation}) and iterative optimisation to provide accurate results of the source. In order to detect the presence of water waves at a detector a ML model (Figure \ref{fig:det-frame}), YOLOv8 \cite{ultralytics_yolov8_nodate}, was able to track wave induced movements of a bob. To affirm the location Monte Carlo simulations (Figures \ref{fig:hist}) quantified the uncertainty in source localisation, and revealing the impact of measurement errors on positional uncertainties. 

Parallels were drawn between the experiment and real GW detectors by comparing the observed wave signals (Figures \ref{fig:d1}-\ref{fig:d3}) to actual GW events such as GW170817 (Figure \ref{fig:NS-GW}) \cite{Abbott_2017}. The shape and detection methodology highlighted similarities in signal analysis, though damping effects in the experiment altered the expected ring-down behaviour. This comparison underscores the project's success in emulating key aspects of GW astronomy while remaining accessible for educational demonstrations.

\section{Limitations and Challenges}
Despite its success, the project faced several limitations as results were effected by environmental factors or flaws in the setup. Friction between the bob and glass tube, as well as viscous damping in water significantly altered the expected wind-up and down phase of the wave (Figures \ref{fig:d1}-\ref{fig:d3}), diverging from real GW signals like GW170817 (Figure \ref{fig:NS-GW}) \cite{Abbott_2017}. While not significantly affecting the overall results of source localisation, the uncertainty in time measurement was altered because resistive forces counteracted the bob's motion as a wave passed the sensor (\ref{sec:4.1.1})

Another limitation of the detector setup was the fact that there were only three sensors as part of the array, this resulted in degeneracy issues where one method for solving the TDoA problem was not enough (Figure \ref{fig:2-loc}). Where with a fourth sensor purely the triangulation method would be sufficient (Figure \ref{fig:4-det}), the source optimisation method was introduced to eliminate this degeneracy instead.

Limitations with the sensor itself attributed to the uncertainties in the measurement of $t_i$. The 30 FPS sampling rate limited the precision of these measurements which introduced discretisation errors where the true value of $t_i$ was masked within the$ 1/30s$ interval between samples of the position of the bob (Figure \ref{fig:d1}). In cases where over 30 FPS are used this uncertainty massively reduces as the position of the bob with respect to time becomes more continuous and the incident time can be accurately measured.

\section{Future Work}
To further both the accuracy and educational value of the demonstration, several improvements are proposed. By implementing matched filtering \cite{Schutz_1999}, alongside using a high speed camera recording at over 30 FPS the system would be better equipped to distinguish true wave signals from noise. The network could also be expanded to four sensors (Figure \ref{fig:4-det}) to eliminate degeneracies without having to rely on optimisation methods. To better replicate GW signals, using freely suspended bobs and lower viscosity fluids will significantly reduce such forces and therefore make for more accurate wave signals by reducing resistive forces. Controlled wave generators could simulate astronomical phenomena similar to that of binary in spiral chirps. By increasing camera sampling rates beyond 30 FPS the time measurement uncertainties would be significantly reduced, with high-speed cameras enabling wavefront tracking to minute fractions of a second. To make the project more catered towards educational demonstrations developing open-source python libraries which build on the existing codebase, it would have an interactive front end and visual demonstrations of the source localisation process, demonstrating the triangulation process (Figure \ref{fig:triangulation}) and Monte Carlo uncertainty analysis (Figure \ref{fig:hist}). These improvements would strengthen both the physical analogy to GW astronomy and the project's utility as an educational tool.