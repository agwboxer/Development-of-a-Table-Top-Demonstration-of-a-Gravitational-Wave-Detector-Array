\font\myfont=cmbx12 at 20pt
\Large
{\myfont Abstract}

\ac{GW} detectors have given humanity insight into the most extreme cosmic events in the universe. They form the foundational backbone of our understanding of black holes and neutron stars. The detection of GWs was ground breaking in itself for proving Einstein's theory of relativity correct. Multiple detectors across the globe are used to confirm the presence of GWs which arrive at earth.

This report explores how multiple observatories collaborate to identify the source location of GWs and, by extension, other cosmic objects. This project aims to develop a tabletop demonstration of a network of GW detectors to explore the principles of source localisation and triangulation. By using multiple water level sensors as analogues for GW detectors, the demonstration will simulate a gravitational wave propagating through a medium (in this instance a water wave) from a singular source. Using video feed from a webcam, a machine learning model is trained to track an object's movement on the water, simulating a GW event. A series of python programs analyse the positional data of the object and infer the position of the source based on \ac{ToA} differences across the network of sensors. 

In the experiment, the distance between sensors is significant relative to the scale, enabling triangulation of the source using ToA differences. In contrast, real-world GW detectors, like LIGO, Virgo, and KAGRA have minuscule baselines compared to cosmic distances, requiring astrophysicists to use wave elements like amplitude and frequency for localisation. The demonstration captures the principle of analysing wave arrival times with a sensor network, offering an accessible exploration of GW detection and source localisation.